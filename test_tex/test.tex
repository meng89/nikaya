\setupinteraction[state=start,focus=standard]

\setupinteractionscreen[option=bookmark]
\enabledirectives[references.bookmarks.preroll]

% 第二个方括号里的放置需要被看到的章节类型。这样理解:需要被看到,所以把上级展开。ps:上级的上级不会被展开。
\placebookmarks[title,subject,subsubject][subsubject]

\setuphead[title][incrementnumber=list]
\setuphead[subject][incrementnumber=list]
\setuphead[subsubject][incrementnumber=list]

\usecolors[crayola]

\define[2]\suttaNumLink{\setupinteraction[color=MidnightBlue, contrastcolor=MidnightBlue]\in{#1}[#2]}


\enabledirectives[backend.pdf.fixhighlight]

\definecolor[pdfhighlight:Hraban][r=.8,g=1,b=1]
\definecolor[pdfhighlight:Hans]  [r=1,g=.8,b=1]
\definecolor[pdfhighlight:Ton]   [r=1,g=1,b=.8]

\setupcomment[
  author=Editor,
  color=pdfhighlight:Editor,
  location=rightmargin,
  symbol=Comment,
]


\starttext
  \starttitle[
      title=\goto{SN 1.1}[url(https://suttacentral.net/SN1.1)]/SN.1.1 \goto{ztxy/blzdgj}[url(https://agama.buddhason.org/SN/SN0001.htm)],
      bookmark=blzdgj,
  ]

    \startsubject[title={\goto{xixihaha}[url(https://github.com/meng89/nikaya)]}, bookmark=666]

      \startsubsubject[title=subsubject1, reference=xxx]
      \stopsubsubject

      \startsubsubject[title=subsubject2, reference=SNx1x1]

      %{\setupinteraction[color=PineGreen]\at{sn 1.1}[xxx]}
      \suttaNumLink{sn_1_1}{SNx1x1}
      %\in{sn }{kk}[SN.1.1]

      \stopsubsubject

    \stopsubject


  \stoptitle

 test \PDFhighlight[Hraban][My comment]{\samplefile{tufte}} test \blank

 test \PDFhighlight[Hans]{what a mess} test \page

 test \PDFhighlight[ ][I would prefer the other one.]{\samplefile{ward}} test \page

Here is an example: by saying


Regular 1
\startcomment
1. This won't be published, but attached as a comment
\stopcomment
Regular 2
\startcomment[commentname]
2. This won't be published, but attached as a comment
\stopcomment
Regular 3
\startcomment[commentname][]
3. This won't be published, but attached as a comment
\stopcomment

\stoptext
