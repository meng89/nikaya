\setupinteraction[state=start,focus=standard]

\setupinteractionscreen[option=bookmark]
\enabledirectives[references.bookmarks.preroll]

% 第二个方括号里的放置需要被看到的章节类型。这样理解:需要被看到,所以把上级展开。ps:上级的上级不会被展开。
\placebookmarks[title,subject,subsubject][subsubject]

\setuphead[title][incrementnumber=list]
\setuphead[subject][incrementnumber=list]
\setuphead[subsubject][incrementnumber=list]

\usecolors[crayola]

\define[2]\suttaNumLink{\setupinteraction[color=MidnightBlue, contrastcolor=MidnightBlue]\in{#1}[#2]}

\starttext
  \starttitle[title=title1]

    \startsubject[title={\goto{xixihaha}[url(https://github.com/meng89/nikaya)]}, bookmark=666]

      \startsubsubject[title=subsubject1, reference=xxx]
      \stopsubsubject

      \startsubsubject[title=subsubject2, reference=SNx1x1]

      %{\setupinteraction[color=PineGreen]\at{sn 1.1}[xxx]}
      \suttaNumLink{sn_1_1}{SNx1x1}
      %\in{sn }{kk}[SN.1.1]

      \stopsubsubject

    \stopsubject


  \stoptitle


\stoptext
