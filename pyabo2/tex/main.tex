\setuppapersize [$size]


\input{setuplayout.tex}


\enableregime[utf]
\mainlanguage[cn]
\language[cn]
\setscript[hanzi] % hyphenation

%%%%%%%
% 字体
\usetypescriptfile[type-imp-myfonts]
\startmode [tc]
  \usetypescript[tcfonts]
  \setupbodyfont[tcfonts,rm,12pt]
\stopmode
\startmode [sc]
  \usetypescript[scfonts]
  \setupbodyfont[scfonts,rm,12pt]
\stopmode
% 字体
%%%%%%%

\usecolors[crayola]
%\usecolors[ral]


%%%%%%%%%%%%
% 页脚页眉
%\setupheader
%  [text]
%  [before={\startframed[frame=off,bottomframe=on,framecolor=black,]},
%   after={\stopframed},
%  ]
%\setupfooter
%  [text]
%  [before={\startframed[frame=off,bottomframe=off,topframe=on,framecolor=black,]},
%  after={\stopframed},
%  ]

%%%%%%%%%%%%%%%%%%%%%%%%%%%%%%%%%%%%%%%%%%%%%%%%%%%%%%%%%%%%%%%%%%%%%%%%%%%%%%%%%%%%%%%%%%%%%%%%%%%%%%%%%%%%%%%%%%%%%%%%
%                                          Header and Footer
%
%\setupheader
%  [text]
%  [before={\startframed[frame=off,bottomframe=on,topframe=off,framecolor=black,]},
%   after={\stopframed},
%  ]
%\setupfooter
%  [text]
%  [before={\startframed[frame=off,bottomframe=off,topframe=on,framecolor=black,]},
%   after={\stopframed},
%  ]


\startsetups mainheaderfooter
\setupheadertexts[]
%\setupheadertexts[{\tf\bf\ss \somenamedheadnumber{chapter}{current}. \getmarking[chapter]}][]
\setupfootertexts[{\tfa \it \pagenumber}]
\stopsetups


\startsetups clearheaderfooter
\setupheadertexts[]
\setupfootertexts[]
\stopsetups
%
%
%%%%%%%%%%%%%%%%%%%%%%%%%%%%%%%%%%%%%%%%%%%%%%%%%%%%%%%%%%%%%%%%%%%%%%%%%%%%%%%%%%%%%%%%%%%%%%%%%%%%%%%%%%%%%%%%%%%%%%%%

% 页脚页眉
%%%%%%%%%%%


%%%%%%%%%%%%%%%
% PDF 属性信息
\setupinteraction[focus=standard] % 重要功能!添加上后才能连接到位置!
\setupinteraction[state=start,style=normal]
\setupinteraction[
    title={$title},
    author={$author},
    keyword={$keyword},
    creator={https://github.com/meng89/nikaya},
]

% PDF 属性信息
%%%%%%%%%%%%%%%


%%%%%%%%%%%%
% PDF书签
%\definelist[tobias]

%\placebookmarks[appendix,part,chapter,section,subsection,tobias][chapter][number=no]

\definelist[mylist]
\placebookmarks[title,subject,subsubject,subsubsubject,subsubsubsubject,subsubsubsubsubject,mylist]
\setuphead[title][incrementnumber=list, style=\tfd\bf\ss, alternative=middle] %\tfd
\setuphead[subject][incrementnumber=list, style=\tfc\bf\ss, alternative=middle]
\setuphead[subsubject][incrementnumber=list, style=\tfb\bf\ss, alternative=middle]
\setuphead[subsubsubject][incrementnumber=list, style=\tfa\bf\ss, alternative=middle]
\setuphead[subsubsubsubject][incrementnumber=list, style=\tfa\bf\ss, alternative=middle]
\setuphead[subsubsubsubsubject][incrementnumber=list, style=\tfa\bf\ss, alternative=middle]


%\setuphead[part]      [number=no,alternative=middle,,placehead=yes,header=empty,footer=empty]


\setupinteractionscreen[option=bookmark]
\enabledirectives[references.bookmarks.preroll]

\enabledirectives[backend.pdf.fixhighlight]



\definecolor[pdfhighlight:莊春江][r=.93,g=.93,b=.93,]



\useexternalfigure[cover][$cover_image]
\defineoverlay[cover][{\externalfigure[cover]}]

\startsetups Background:cover
\setupbackgrounds[page][background=cover]
\stopsetups

\startsetups Background:off
% 只要把 background 留空或这设置成无效的名字,就会没有背景
\setupbackgrounds[page][background=,]
\stopsetups

%\showframe

%%%%%%%%%%%%%%%%%%%%%%%%%%%%%%%%%%%%%%%%%%%%%%%%%%%%%%%%%%%%%%%%%%%%%%%%%%%%%%%%%%%%%%%%%%%%%%%%%%%%%%%%%%%%%%%%%%%%%%%%

\startdocument
\setups[clearheaderfooter]
%\setuppagenumber[number=1]

\setups[Background:cover]
\bookmark[mylist]{封面}
\null
\page
\setups[Background:off]

\input{fanli.tex}

\input{homage.tex}

\setups[mainheaderfooter]

\input{suttas.tex}

\setups[clearheaderfooter]

\input{readme.tex}

\stopdocument
